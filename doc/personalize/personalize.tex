\documentclass[11pt]{article}

\usepackage{fontspec}
\setmainfont{AR PL UKai CN}

\usepackage{verbatim}
\usepackage{amstext}
\usepackage{indentfirst}
\usepackage[left=1.5in,top=1in,right=1.5in,bottom=1in]{geometry}

\begin{document}

\title{个性化阅读升级}
\author{cqy}
\maketitle

%
\section{背景}
\subsection{升级目标}
使内容分发系统成为满足用户不同阅读需求的全面个性化系统
\subsection{现状}
\begin{enumerate}
\item 文章来源以知名权威站点为主
\item 分发系统本身没有考虑多元化需求
\item 没有根据兴趣来进行排版,不能体现内容聚合的优势
\item 兴趣tag不准,很多tag不是用户兴趣点,没有结构化
\item 用户理解比较单薄,没有对用户的需求进行挖掘、推断
\item 模型在时效性、实时特征、探索算法方面没有开展工作
\end{enumerate}

%
\section{需求分析}
\subsection{六个基本层次}
\begin{enumerate}
\item 生理刺激
\item 娱乐休闲
\item 信息获取
\item 知识获取
\item 审美情趣
\item 思想进步
\end{enumerate}

以上是阅读产品主要满足的六大基本需求层次,由低到高反映了人们不同的文化程度、修养状态和价值取向。

\subsection{需求和语义的关系}
目前的标签体系主要是以语义来建立的,是两个不同的维度。部分语义标签就是需求,但是不能完全涵盖需求;频道的划分就是按需求划分的,但频道不都是标签。同一篇内容对不同用户有不同的需求。

\subsection{需求和兴趣的关系}
在内容聚合平台中,需求就是兴趣。

\subsection{细分的需求}
太多太复杂,不可穷举,只能按受众价值和产品定位的优先级去做。

\subsection{分析结论}
\subsubsection{内容聚合能满足的需求}
生理型需求、娱乐休闲型需求(网络热点、花边报道、软新闻)、信息获取型需求(软、硬新闻、本地新闻)、知识获取型需求、部分社交型需求(非阅读需求)
\subsubsection{内容聚合不能满足的需求}
深度知识需求(教育、科研)、审美情趣需求、
思想进步需求(通过深度阅读+冥想)

\subsubsection{总结}
大部分用户不在乎是人工推荐还是机器推荐,也不知道自己想看什么内容。但用户知道自己不喜欢看什么,一旦觉得不好看,用户会做出实际行动选择更好的替代品。
所有用户都有多层次的阅读需求,内容平台要特别注意考虑满足同一用户高层次和低层次的阅读需求。
\begin{eqnarray}
\text{频道} \subseteq \text{需求} = \text{兴趣标签} \neq \text{语义标签}
\end{eqnarray}

%
\section{体验分析}

\subsection{中断}
\subsection{负反馈}
\subsection{心理体验}
\subsubsection{兴趣转移}
\subsubsection{兴趣厌烦}

%
\section{项目分解}
\subsection{信息抽取}
\subsection{文章流转机制}
\subsection{兴趣轮换机制}
\subsection{实时人群推断}
\subsection{需求预估}
\subsection{模型}
\subsection{排序}

\begin{comment}
$$\sum_{p\rm\;prime}f(p) = \int_{t>1}f(t)d\pi(t).$$

$$2\uparrow\uparrow k
\mathrel{\mathop=^{\rm definition}}
2^{2^{\cdot^{\cdot^{\cdot^2}}}}
\vbox{\hbox{$\Big\}\scriptstyle k$}\kern0pt}.$$
\end{comment}

\end{document}